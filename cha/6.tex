\chapter{自旋}
\section{考试要求}
\begin{enumerate}
    \item 了解斯特恩一盖拉赫实验。电子自旋回转磁比率与轨道回转磁比率。
    \item 熟练掌握自旋算符的对易关系和自旋算符的矩阵形式 (泡利矩阵)、与自旋相联系的 测量值、概率和平均值等的计算以及其本征值方程和本征矢的求解方法。
    \item 了解电磁场中的薛定谔方程和简单塞曼效应的物理机制。
    \item 了解自旋-轨道藕合的概念、总角动量本征态的求解及碱金属原子光谱的精细和超精 细结构。
    \item 熟练掌握自旋单态与三重态求解方法及物理意义, 了解自旋纠缠态概念。
\end{enumerate}
\section{基础知识梳理}
\subsection{斯特恩——盖拉赫实验}
\subsection{电子的自旋}
\paragraph*{自旋波函数与自旋态}
电子的自旋波函数是旋量波函数:$$\psi(\mathbf{r},s_z)=
\begin{pmatrix}
    \psi\left(\mathbf{r},\dfrac{\hbar}{2}\right)  \\
    \psi\left(\mathbf{r},\dfrac{-\hbar}{2}\right) 
\end{pmatrix}=\phi(\mathbf{r})\chi(s_z)
    $$
    $s_z=\pm \dfrac{\hbar}{2}$的本征态为$\alpha=\chi_{1/2}(s_z)=\begin{pmatrix}
        1\\
        0
    \end{pmatrix}=\ket{\uparrow}$和$\beta=\chi_{-1/2}(s_z)=\begin{pmatrix}
        0\\
        1
    \end{pmatrix}=\ket{\downarrow}$

    一般的电子自旋态可以表示为两者的线性叠加:
    $$\chi(s_z)=\begin{pmatrix}
        a\\
        b
    \end{pmatrix}=a\alpha+b\beta=a\ket{\uparrow}+b\ket{\downarrow}$$
    \paragraph*{自旋算符与泡利算符}
    自旋算符$\hat{\mathbf{S}}$的定义式为:
    \begin{align}
        \hat{\mathbf{S}}\times\hat{\mathbf{S}}=i\hbar\hat{\mathbf{S}}
    \end{align}
    引入泡利算符$\hat{\sigma}$,与自旋算符满足如下关系:
    \begin{align}
        \hat{\mathbf{S}}=\dfrac{\hbar}{2} \hat{\sigma}
    \end{align}
