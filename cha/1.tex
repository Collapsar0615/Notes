\chapter{狭义相对论}
\section{狭义相对论的实验基础}
\subsection{参考系问题}
电磁现象的基本规律是Maxwell方程组和外电场洛伦兹力作用下带电粒子的牛顿第二定律.在真空中,分别表述为:
\begin{align}\label{eq1}
    \left\{
        \begin{aligned}
            \nabla \cdot \boldsymbol{E}&=\dfrac{\rho}{\varepsilon_0}\\
            \nabla \times \boldsymbol{E}&=-\dfrac{\partial \boldsymbol{B}}{\partial t}\\
            \nabla \cdot \boldsymbol{B}&=0\\
            \nabla \times \boldsymbol{B}&=\mu_0 \boldsymbol{j}+\varepsilon_0 \mu_0\dfrac{\partial \boldsymbol{E}}{\partial t}
        \end{aligned}
    \right.
\end{align}
\begin{align}
\dfrac{d\boldsymbol{p}}{dt}=q\boldsymbol{E}+q\boldsymbol{u}\times \boldsymbol{B}
\end{align}
在真空中远离电荷电流的区域,Maxwell方程组\ref{eq1}中$\rho,\boldsymbol{j}=0,$由此出发可以得到电磁场的波动方程:
\begin{align}
\nabla ^2 \begin{pmatrix} \boldsymbol{E} \\ \boldsymbol{B} \end{pmatrix}
-\dfrac{1}{c^2}\dfrac{\partial ^2}{\partial t^2}   \begin{pmatrix} \boldsymbol{E} \\ \boldsymbol{B} \end{pmatrix} =0
\end{align}
其中,$c=\dfrac{1}{\sqrt{\mu_0\varepsilon_0}}\approx 3 \times 10^8 m/s$是真空电磁波的传播速度.
\subsection{实验基础}
\section{狭义相对论的基本原理}
\section{狭义相对论的时空结构}
\section{闵式空间中的张量}
\section{狭义相对论中的加速参考系}